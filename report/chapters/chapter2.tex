\chapter{Background Research \& Notation}
\label{chapter2}
\section{Markov Decision Processes}
The Markov Property states that the future is conditionally independent of the past given the present. A sequential decision-making problem that satisfies the Markov Property is known as a Markov Decision Problem, and can be modelled by a Markov Decision Process (MDP) \citep{10.5555/528623}. Where an agent is able to fully observe its state, the problem can be modelled as an MDP. Conversely, where an agent can only partially observe its state, the problem can be modelled by a Partially Observable Markov Decision Process (POMDP).
Consider an agent playing a perfect information game, such as Chess or Go, the agent (and its opponent) are able to fully observe the state of the game with ease. However, a robot navigating through a maze may not be able to observe its exact state due to uncertainty in its sensors.
Furthermore, an MDP can be stationary or non-stationary which refer to whether its dynamics stay the same, or change temporally.
An MDP that contains absorbing states, that is any state that upon entering the task terminates, then it is said to be episodic.
\\Within this work we assume that the agent is fully able to observe its state, the environment is stationary and can be discretised in some way and tasks are episodic.
% and tasks are episodic (there exist some terminal states which indicate the end of a task); 
Hence we consider \textbf{stationary}, \textbf{finite}, \textbf{undiscounted} MDPs.
Thus, an MDP is a 5-tuple: $\text{MDP} = (S,A,T,R)$ where:
\begin{itemize}
    \item $S$ is a finite set of states.
    \item $A$ is a finite set of actions.
    \item $T : S \times A \times S \rightarrow [0,1]$ is the transition function, which determines the probability of transitioning from a state $s \in S$ to $s' \in S$ with an action $a \in A$.
    \item $R:S \times A \times S \rightarrow \mathbb{R}$ is the reward function, which determines the reward signal, $r \in \mathbb{R}$ received by the agent from transitioning from a state $s \in S$ to $s' \in S$ with the action $a \in A$. This reward is extrinsic to the agent; it comes from the environment.
\end{itemize}
\begin{defn}
\label{defn:determinism}
An MDP is said to be deterministic if $\forall s,s' \in S, \forall a \in A$, $T(s,a,'s) \in \{0,1\}$.
\end{defn}
\begin{defn}
\label{defn:stochastic}
An MDP is said to be stochastic if it is not deterministic
\end{defn}
% and $\forall s,s' \in S, \forall a \in A$, $T(s,a,'s) \in [0,1]$
% \end{defn}
% The transition function, $T$ is a key indicator about the nature of the environment.
% If $\forall s,s' \in S, \forall a \in A$, $T(s,a,'s) \in \{0,1\}$, then the environment is deterministic, otherwise it is stochastic.
% A deterministic environment is one which there is no variance in the outcomes of action in a given state; taking the same action in the same state always produces the same outcome. Whereas a stochastic (or non-deterministic) environment has uncertainty associated with transitions.
\subsection{Policies, Value and Quality}
A policy, denoted by $\pi$, is a mapping of states to actions that describes a behaviour within an MDP. The desired behaviour within an MDP is to maximise the cumulative reward, this is known as the optimal policy and denoted $\pi^*$. The cumulative reward is given by:
\begin{equation}
\label{eqn:return}
G_t = \sum_{k=t+1}^TR_{k}
\end{equation}
which represents the cumulative reward received from time $t$ to some finite time $T$.
A Value Function, or state-value function, denoted $V^\pi(s)$, measures the expected cumulative reward that can be received starting in each state $s \in S$ and following a policy $\pi$:
\begin{equation}
\label{eqn:vs}
    V^\pi(s) = \mathbb{E}^\pi\Bigg[G_t | s_t = s\Bigg] = \mathbb{E}^\pi\ \Bigg[\sum_{k=t+1}^TR_{k} | s_t = s \Bigg]
\end{equation}
Where $\mathbb{E}$ represents the expected value that the agent follows $\pi$, $t$ is an arbitrary time step and $T$ is a finite time horizon.
The optimal Value Function, $V^*$, is the one that is maximum for every $s \in S$ and is given by:
\begin{equation}
\label{eqn:vsm}
     V^*(s) = \max_\pi V^\pi(s) 
\end{equation}
This can also be represented in a recursive form, known as the Bellman Optimality equation for $V$:
\begin{equation}
\label{eqn:vsB}
V^*(s) = \max_a\sum_{s'}T(s,a,s')[R(s,a,s')+V^*(s')]
\end{equation}
Similarly, a Q-Function, or action-value function, denoted $Q^\pi(s,a)$, measures the expected cumulative reward that can be received starting in each state $s \in S$, taking an action $a \in A$ and thereafter following a policy $\pi$:
\begin{equation}
\label{eqn:qsa}
Q^\pi(s,a) = \mathbb{E}^\pi\Bigg[G_t | s_t = s,a_t = a\Bigg] = \mathbb{E}^\pi\Bigg[\sum_{k=t+1}^TR_{k}|s_t=a, a_t = a\Bigg]
\end{equation}
Where $\mathbb{E}$ represents the expected value that the agent follows $\pi$, $t$ is an arbitrary time step and $T$ is a finite time horizon.
The optimal Q-Function, $Q^*$, is the one that is maximum for every $s,a \in S \times A$ and is given by:
\begin{equation}
\label{eqn:qso}
Q^*(s,a) = \max_\pi Q_\pi(s,a)
\end{equation}
Similarly to the Value Function, this can also be represented in a recursive form, known as the Bellman Optimality equation for $Q$:
\begin{equation}
\label{eqn:qsB}
Q^*(s,a) = \sum_{s'}T(s,a,s')[R(s,a,s')+\max_{a'}Q^*(s',a')]
\end{equation}
An interesting observation is that $\pi^*$ can be derived from both $V^*$ and $Q^*$. In the case of $V^*$, $\pi^*$ can be derived by determining at each state, which actions yield the next state with the maximum value. In the case of $Q^*$, $\pi^*$ can be derived by simply taking the action with the highest Q-value at each state \cite{Sutton1998}.

% The optimal value function, $V^*(s)$, is the one that is maximum for every $s \in S$, it also happens to correspond to $\pi^*$.
% Similarly, a Q-Function, denoted $Q^\pi(s,a)$, measures the expected cumulative reward that can be received starting in each state $s \in S$ and taking an action $a \in A$ and thereafter following $\pi$. The optimal Q-function, $Q^*(s,a)$, is the one that is maximum for every $s \in S, a \in A$, this also corresponds to $\pi^*$. Therefore, $\pi^*$ can be determined by discovering $V^*$ or $Q^*$.
% \subsection{Optimality}
% The optimality equation for a Value Function, $V$ is given as such:
% \begin{equation}
% \label{eqn:vstar}
%     V^*(s) = \max_a\sum_{s'}T(s,a,s')[R(s,a,s')+V*(s')]
% \end{equation}
% The optimality equation for a Q-Function, $Q$ is given as such:
% \begin{equation}
% \label{eqn:qstar}
% Q^*(s,a) = \sum_{s'}T(s,a,s')[R(s,a,s')+\max_{a'}Q^*(s',a')]
% \end{equation}
% The Bellman Equation determines the expected reward for being in a state $s \in S$ and following a fixed policy $\pi$:
% $$V^\pi(s) = R(s, \pi(s)) + \sum_{s'}T(s'|s,\pi(s))V^\pi(s')$$
% Where $V^\pi$ is the value function of the policy, $\pi$.\\
% The Bellman Optimality equation determines the reward for taking the action giving the highest expected return.
% $$V^{\pi*}(s) = \text{argmax}_a{R(s,a) + \sum_{s'}P(s'|s,a)V^{\pi*}(s')}$$
% Where $V^{\pi*}$ is the value function of the optimal policy.

\section{Planning}
Planning involves reasoning on a model of an environment in order to produce a sequence of actions that will achieve a specified goal whilst satisfying some constraints or achieving some objectives, such as minimising cost \cite{DBLP:books/aw/RN2020, Lav06, GhallabNauTraverso04}. 
Planning has been a widely studied topic in AI for many years, as such there exist many planning methods and algorithms. However, although planning is central to this work, particular planning methods and algorithms are not; therefore, we only consider heuristic search methods and planning by Dynamic Programming.

% plannning methods and algorithms are not the main topic of this work
%Planning has been a widely studied topic in AI for many years, as such there exist many planning methods. Some classical examples are STRIPS \cite{fikes1972strips} and SATPLAN \cite{kautz}. Heuristic search methods exists such as A* \cite{4082128}. In the context of MDPs, planning can also be done by dynamic programming 


% from propositional logic based planning, such as STRIPS \cite{fikes1972strips}, to planning by satisfiability, such as SATPLAN \cite{kautz}, to heurstic search methods, such as A* \cite{4082128}, to Dynamic programming methods, such as Value Iteraton \cite{
% Planning algorithms range vastly; from classical methods such as STRIPS, 


% Planning can be as simple as using an  heuristic search algorithm, such as A* \cite{4082128}, to reason a sequence of actions, or as complex as solving an MDP by Dynamic Programming and sampling a plan from the optimal policy estimate.
\subsection{Heuristic Search}
Heuristic search (or informed search) is a class of search algorithms that use problem specific knowledge in the form of heuristics to guide the search. A heuristic, denoted $h(s)$, is a function that estimates the cost of transitioning from the current state, $s$, to the goal state. We consider best-first search algorithms, which aim at expanding states that appear best according to an evaluation function, denoted $f(s)$. 
% Heuristic search uses heuristic functions to guide the exploration of possible solutions. A heuristic is a function that estimates the cost of transitioning from the current state to the goal state. The choice of heuristic can vary, depending on the problem; and in some domains it can be impossible to design a good heuristic.
\subsubsection{A* Search}
\label{sec:astar}
A* Search (or just A*) \cite{4082128} is a heuristic search algorithm that is a form of best-first search. A cost function, $g(s)$, is introduced, which indicates the cost to get from the start state to a state $s$. The main idea is that the evaluation function can be a combination of the cost to get to the state being evaluated, $g(s)$, and the estimated cost to get from the state being evaluated to the goal state, thus $f(s) = g(s) + h(s)$ and indicates the estimated cost of the minimum cost solution that passes through $s$ \cite{DBLP:books/aw/RN2020}. A* works by guiding the search to expand states with the lowest value for $f(s)$ first, in order to determine a sequence of states which is the path (or in our case, the plan). A* is a complete algorithm; it will always find a solution if one exists. However, the optimality of A* depends on the choice of the heuristic: the heuristic must be admissible, which means that it must never overestimate the cost of getting from the current state to the goal state; in this sense, the heuristic must always be optimistic.
\\
% The idea is that the evaluation function can be a combination of the cost to reach the state, $s$, being evaluated, denoted $g(s)$, and the estimated cost to get from 
% A* search is a heuristic search algorithm, which is limited to discrete state and action spaces. It can be difficult to choose the heuristic, the heuristic must be admissable \cite{DBLP:books/aw/RN2020}; this means that it must always underestimate the cost of getting from the current state to the goal, and never overestimate it. An interesting extension of this is LRTA* \cite{KORF1990189}, which is akin to model-based RL.
\subsection{Dynamic Programming}
Dynamic Programming (DP) \cite{Bellman:1957, DBLP:books/lib/Bertsekas05} can be used to solve an MDP by computing an optimal policy with respect to it. It's important to note that the optimal policy is optimal with respect to model.
The two main algorithms that we consider are Policy Iteration \cite{Bellman:1957, howard:dp} and Value Iteration \cite{Bellman:1957}; both of which are forms of Generalised Policy Iteration (GPI). GPI refers to any process that involves Policy Evaluation and Policy Improvement acting with each other. As it happens, most DP and RL methods are forms of GPI \cite{Sutton1998}.


% Dynamic Programming (DP) \cite{Bellman:1957, DBLP:books/lib/Bertsekas05} can be used to compute an optimal policy given a perfect model of the environment, embedded in an MDP \cite{Sutton1998}. If the model is not perfect, then the computed policy is only optimal for the model, not for the real environment; however this can still be useful. 


% Dynamic Programming \cite{Bellman:1957, DBLP:books/lib/Bertsekas05} is a problem-solving method which involves breaking down problems into smaller sub-problems, solving them, storing the solutions, and combining them to solve the original problem. Given a perfect model of the environment, embedded in an MDP, an optimal policy can be computed using Dynamic Programming, through both Policy Iteration  and Value Iteration.
\subsubsection{Policy Evaluation and Improvement}
Given a policy, $\pi$, it can be evaluated by computing its Value Function, $V^\pi$:
\begin{equation}
\label{eqn:policyeval}
V^\pi = \sum_a \pi(a|s)\sum_{s'}T(s,a,s')[R(s,a,s') + V^\pi(s')]
\end{equation}
Policy Improvement is the process of generating an improved policy from a suboptimal policy \cite{DBLP:books/lib/Bertsekas05}. A policy $\pi$ could be improved by contemplating taking an action $a \in A$ in a state $s \in S$, such that $\pi(s) \neq a$ and then continue following $\pi$ thereafter. We can evaluate this change by computing the Q-function for $s,a$:
\begin{equation}
\label{eqn:qval}
Q^\pi(s,a) = \sum_{s'}T(s,a,s')[R(s,a,s') + V^\pi(s')]
\end{equation}
Then comparing $Q^\pi(s,a)$ with $V^\pi(s)$. If $Q^\pi(s,a) > V(s)$, then the policy is update and improved as such: $\pi(s) = a$.
% Policy Improvement \cite{Bellman:1957} is the process of generating an improved policy from a suboptimal policy \cite{DBLP:books/lib/Bertsekas05}. One way to improve a policy $\pi$ is to contemplate taking an action $a$ in a state $s$, such that $\pi(s) \neq a$ and then continue following $\pi$ afterwards. If this change turns out to be better, then $\pi(s) \leftarrow a$; the policy has been improved.
\subsubsection{Policy Iteration}
Policy Iteration (PI) \cite{Bellman:1957, howard:dp} is a method for computing the optimal policy of an MDP. Assuming an arbitrary initial policy, $\pi$, at each step $\pi$ is evaluated, yielding a Value Function $V^\pi$ (or a Q-Function $Q^\pi$). The policy is then updated greedily with respect to $V^\pi$ (or $Q^\pi$). This iterative process leads to monotonic improvements to $\pi$.
\subsubsection{Value Iteration}
Value Iteration (VI) \cite{Bellman:1957} is a method for computing the optimal estimate for the Value or Q-Function of an MDP. As the estimate approaches optimality, the policy that is greedy with respect to the Value or Q-Function also approaches optimality \cite{series/synthesis/2010Szepesvari}.
In the context of Value Functions, assuming an arbitrary initial estimate, $V$, at each step $V$ is updated as such:
\begin{equation}
\label{eqn:vupdate}
V(s) = \max_a\sum_{s'}T(s, a, s')[R(s, a, s')+V(s')]
\end{equation}
for all $s \in S$. This update originates from the Bellman optimality equation for $V$, Equation \ref{eqn:vsB}.
For Q-Functions, assuming an arbitrary initial estimate, $Q$, at each step $Q$ is updated as such:
\begin{equation}
\label{eqn:qupdate}
Q(s,a) = \sum_{s'}T(s,a,s')[R(s,a,s') + \max_{a'}Q(s',a')]
\end{equation}
for all $s \in S$, $a \in A$. This update originates from the Bellman optimality equation for $Q$, Equation \ref{eqn:qsB}.
% Assuming an arbitrary estimate of the Value Function, $V$, or the Q-Function, $Q$, at each step
% Value Iteration (VI) is a method for computing the optimal policy in an MDP; this policy is akin to a plan and can be used in the same way. Assuming an arbitrary estimate of the Value Function, $V$, at each step, $V$ is updated as such:
% $$V(s) = \max_a\sum_{s'}T(s, a, s')[R(s, a, s')+V(s')]$$
% VI 
\section{Reinforcement Learning}
Within an RL setting, formalised by an MDP, an agent learns how to behave in an environment by interacting with it through actions, at discrete, sequential, time steps, and observing the affects through its new state and a scalar reward signal, as seen in Figure \ref{fig:rl}. The reward signal may be delayed, meaning that the consequences of actions may not be known until long after they are taken \cite{barto1990learning}. This gives rise to the (temporal) credit assignment problem \cite{Minsky:1961:ire}, the problem of determining which actions led to an outcome and assigning credit among them; it's often the case that a sequence of actions led to an outcome, rather than a single action.
The goal of the agent is to learn a policy, $\pi^*$, that maximises the expected cumulative long-term reward \cite{Sutton1998}.
\begin{figure}[h!]
    \centering
    \includegraphics[max size={300}{300}]{report/assets/rl.png}
    \caption{The RL Loop}
    \label{fig:rl}
\end{figure}
\subsection{Model-Free Learning}
Model-free (or direct) RL is the traditional instantiation of RL, where the agent learns a policy directly from experience gained by interacting with the environment; it is purely trial-and-error. Model-free learning tends to be quite flexible to varying problems, since no assumptions are made about the environment's dynamics. Furthermore, since the environment's dynamics do not need to be stored, learned or considered, model-free learning is scalable and does not suffer from the \textit{curse of dimensionality} (in the non-tabular cases) and can be computationally efficient, as there is generally no lookahead deliberation process involved. However, model-free learning can be sample-inefficient; the agent may need to take many actions before discovering the optimal policy.
\subsubsection{Temporal Difference Learning}
Temporal Difference (TD) Learning \cite{10.5555/911176, 5392560, 5391906} aims at solving the problem of temporal credit assignment by combining ideas from Monte Carlo methods, which learn directly from experience in the environment, and Dynamic Programming methods, which bootstrap estimates from other previously learned estimates. It is a form of Generalised Policy Iteration, in that it aims to produce an optimal estimate of the Value Function, $V_\pi^*$, starting from an initial estimate, $V_\pi$, for a given policy, $\pi$,\cite{Sutton1998}.
\\The core idea of TD Learning is to update the estimate of the Value Function, $V_\pi$, whenever there is a difference between temporally successive predictions \cite{Sutton:1988}. This difference is known as the TD error. The updated estimate is bootstrapped from the previous estimate, meaning that TD learning essentially learns a prediction from another prediction.
\\The simplest TD Learning algorithm is TD(0) \cite{Sutton:1988}, which updates it's estimates as such:
\begin{equation}
\label{eqn:td0update}
V(s_t) = V(s_t) + \alpha[r_{t+1} + V(s_{t+1}) - V(s_t)]
\end{equation}
The TD target is the current prediction of the expected cumulative reward: $r_{t+1} + V(s_{t+1})$. The TD error is $r_{t+1} + V(s_{t+1}) - V(s_t)$.
$0 \le \alpha \le 1$ is the learning rate, which is used to control how much weight is given to the TD error when updating the estimate. 
\\TD Learning can also be extended to learn a Q-Function, which stores the value of state-action pairs rather than the value of states. For each state-action pair, it stores an estimate of the expected cumulative reward starting in that state and taking an action, and following a fixed policy.
% \\Starting from an initial estimate of the Value Function, $V_\pi$, for the current policy being followed, $\pi$, the goal of TD Learning is 
% \\The core idea of TD learning is to learn the optimal estimate of the Value Function of the current policy being followed. The estimate is updated whenever there is a difference between the expected reward and actual reward received at each discrete time step; learning occurs whenever there is a difference between temporally successive predictions.
% The core idea of TD learning is to maintain an estimate of the Value Function of the current policy being followed, this estimate is updated whenever there is a difference between temporally successive predictions; this difference is known as the TD error.
% TD Learning algorithms can be on-policy, such as SARSA \cite{rummery:cuedtr94}, where the value of the policy being currently carried out by the agent is learnt, or off-policy, such as Q-Learning \cite{Watkins:1989, journals/ml/WatkinsD92}, where the value of the optimal policy is learnt independently of the agent's actions following the current policy.
% \cite{PooleMackworth17}.


% Within Temporal Difference (TD) \cite{10.5555/911176, 5392560, 5391906}, learning is driven by the error/difference between temporally successive predictions, so learning occurs whenever there is a change in prediction over time. It's a method for learning to predict; learning a prediction from another later learned prediction.
% The problem of temporal credit assignment gives rise to Temporal Difference (TD) Learning \cite{10.5555/911176, 5392560, 5391906}. TD Learning maintains an estimate of the Value Function, which is updated whenever there is a change in prediction over time.
\paragraph*{Q-Learning} \cite{Watkins:1989, journals/ml/WatkinsD92} is an off-policy TD Learning method that learns a Q-function. It is off-policy, as the update rule assumes a greedy policy - this means that the value of the optimal policy is learned independently of the agent's actions following the current policy. Q-values are iteratively updated using the Bellman equation, the update rule is as follows:
\begin{equation}
\label{eqn:qlearningupdate}
Q(s_t,a_t) \leftarrow Q(s_t,a_t) + \alpha[r_{t+1} + \max_aQ(s_{t+1}, a) -Q(s_t,a_t)]
\end{equation}
Where $0 \le \alpha \le 1$ is the learning rate, which indicates how quickly learning occurs.
% \\Clearly Q-Learning comes from Dynamic Programming, and in that sense it is a tabular method; a Q-table is maintained which stores the Q-values for each state-action pair. For this reason, it doesn't scale too well to large state/action spaces - however it is suitable for the domains that we consider within this work.
% \\The result of Q-Learning is that a deterministic, greedy policy is learned.
\paragraph*{SARSA} (State Action Reward State Action) \cite{rummery:cuedtr94} is an on-policy TD Learning method that learns a Q-function. It is on-policy, as the update rule assumes that the current policy continues being followed - this means that the value of the current policy being followed is learned. Q-values are iteratively updated using the Bellman equation, the update rule is as follows:
\begin{equation}
\label{eqn:sarsaupdate}
Q(s_t, a_t) \leftarrow Q(s_t, a_t) + \alpha[r_{t+1} + Q(s_{t+1}, a_{t+1})-Q(s_t, a_t)]
\end{equation}
Where $0 \le \alpha \le 1$ is the learning rate, which indicates how quickly learning occurs.
\subsection{Model-Based Learning}
% We define Model-based (or indirect) RL as any algorithm that aims to maximise a numerical reward signal and does 
Model-based (or indirect) RL is where an agent uses a known or learned model of the environments dynamics (in the form of an MDP) in order to learn the optimal policy \cite{MAL-086}. The agent may learn the model and policy at the same time, as in the Dyna family \cite{Sutton:1990, 10.1145/122344.122377} or learn a policy by planning over a known model, as in AlphaZero \cite{DBLP:journals/corr/abs-1712-01815} or learn a policy by planning over a learned model, as in MuZero \cite{DBLP:journals/corr/abs-1911-08265}.
% Model-based learning tends to provide better sample efficiency than model-free approaches, since learning can occur from simulation rather than directly from experience \cite{RLSOTA11} and the number of exploratory actions can be reduced by using targeted exploration \cite{Thrun-1992-15850}. However, this comes at a computational cost; planning can be expensive, especially if the state and action spaces are large.
% \\Comparison
% \begin{itemize}
%     \item model-free RL is more scalable, simpler to program
%     \item model-based RL can be more efficient, can overcome delayed rewards, however comes at a cost: planning takes time and computational power.
% \end{itemize}
\subsubsection{Model Learning}
Model Learning can be a difficult problem due to uncertainty due to limited data, stochasticity and partial observability. We omit the problem of partial observability here, since we are not considering POMDPs, however stochasticity and uncertainty are very relevant to us. Stochasticity and uncertainty can be overcome through sufficient exploration; sampling a transition multiple times can give a better estimate of the probability distribution which reduces uncertainty.
\\Model learning can be viewed as a supervised learning problem \cite{JORDAN1992307}. In discrete environments, exact models of the environment's dynamics can be learned in the form of a \textit{tabular maximum likelihood model} \cite{10.1145/122344.122377} which  maintains a table with an entry for every possible transition.
In the stochastic case, for each transition the table will store:
\begin{equation}
\label{eqn:tmlmupdate}
T(s, a, s') = \frac{n(s, a, s')}{\sum_{s'}n(s,a,s')}
\end{equation}
Where $n(s,a,s')$ represents the number of times the transition has been observed.
% The same can be used for the deterministic case, which will result in all transitions mapping to either 0 or 1. It should be noted that an environment could be discretised in order to learn an exact model, but in-fact this model will be approximate due to this discretisation.
A key drawback to this approach is the lack of scalability to large state spaces.
\\In continuous domains an exact model cannot be learned, due to the infinite number of states (and potentially actions). An approximate model can be learned by using state aggregation (which discretises the state space) and a tabular maximum likelihood model \cite{Kuvayev1996ModelBasedRL}.
However, more commonly an approximate model is learned through Function Approximation methods such as Linear Regression \cite{DBLP:journals/corr/abs-1206-3285, NIPS2007_b7bb35b9} and Gaussian Processes \cite{10.5555/3104482.3104541}, which enables scaling to continuous domains and large state spaces.
% % However, as with all tabular methods, this doesn't scale well.
% A key drawback of this approach is the lack of scalability to large state and action spaces.
% In continuous environments, tabular approaches are not possible due to the infinite number of states and actions. However, a continuous environment's dynamics can be approximately learned using Function Approximation.


% The agent learns to act in the environment by either learning or being provided with a representation of the dynamics of the environment.
% Model-based RL is where the agent learns to act in an environment, and has some understanding of the dynamics of the environment in the form of a model. By the nature of models, the model is inaccurate, more often than not. \cite{Sutton:1990, MAL-086, 10.1145/122344.122377, Kuvayev1996ModelBasedRL, RLSOTA11}


% \section{Exploration in Reinforcement Learning}
% % There is not a definitive taxonomy of exploration methods in RL.
% \subsection{Model-free}
% Purely random exploration comes in the form of a \textbf{Random Walk}, or unguided random search \cite{anderson86}, which arises from randomly sampling actions with uniform probability, as seen in Equation \ref{eqn:rw}. 
% \begin{equation}
% \label{eqn:rw}
%     a_t \sim P(A)
% \end{equation}
% Where $P(A)$ is the uniform distribution over the action space, $A$. Exploration occurs naturally when the agent moves away from the goal, rather than closer to it, due to the uniform selection of actions. This is perhaps the most naive method of exploration, since it is entirely random and is therefore very inefficient.
% \\ \textbf{Boltzmann exploration} samples an action according to the Boltzmann distribution:
% \begin{equation}
% \label{eqn:boltzmann}
% \pi(a|s) = \frac{e^{Q(s,a)/\tau}}{\sum_{a' \in A}e^{Q(s,a')/\tau}}
% \end{equation}
% Where $\tau$ is the temperature, and determines how frequently random actions are chosen. As $\tau$ decreases, the policy approaches greediness \cite{DBLP:journals/corr/cs-AI-9605103, DBLP:journals/corr/abs-2109-00157}. $\tau$ may be decayed temporally, in the same fashion as $\epsilon$ can be decayed temporally in $\epsilon$-greedy. Here we gave the definition in the context of a $Q$-function, however this can be replaced with any value that estimates the reward received from taking action $a$. The main drawback of this approach is that the initial value for $\tau$ needs to be carefully selected, and may need to be tuned for each environment. This is because as $\tau$ approaches 0 the agent acts greedily, whereas at large values of $\tau$, all actions have an approximately uniform probability of being selected leading to exploration akin to a Random Walk \cite{DBLP:journals/corr/abs-2109-00157}.
% \\\textbf{$\\epsilon$-greedy} \cite{Watkins:1989, conf/nips/Sutton95} uses a hyperparameter, $\epsilon$ to balance between exploration and exploitation. With probability $\epsilon$ the agent explores by taking a random action, which is sampled with uniform probability, with probability $1-\epsilon$ the agent exploits by taking the best action, which is selected greedily with respect to the current policy; this is summarised in Equation \ref{eqn:egreedy}.
% \begin{equation}
% \label{eqn:egreedy}
% a_t = 
% \begin{cases}
% \pi(s_t) \qquad \qquad \qquad \text{with probability} \quad 1-\epsilon\\
% \text{random action} \qquad  \text{with probability} \quad $\epsilon$
% \end{cases}
% \end{equation}
% Where $0 \le \epsilon \le 1$, and $\pi$ is the current greedy policy.
% Whilst this provides a clear method for balancing exploration and exploitation, it lacks temporal persistence; although $\epsilon z$-greedy \cite{dabney2021temporallyextended} offered an extension to $\epsilon$-greedy that does use temporal persistance through temporally-extended sequences of actions. Furthermore, by the nature of the name, it is a greedy method which could lead to a local optima. $\epsilon$ may be decayed temporally, to ensure that lots of exploration happens early on, and then exploitation occurs more when learning has occurred. Whilst $\epsilon$-greedy is very easy to implement, it is very inefficient and only converges to optimality in the limit, under certain conditions.
% \subsection{Optimistic}
% \\Explicit Explore or Exploit ($E^3$) \cite{Kearns+Singh:2002} maintains a partial model of the environment through collecting statistics, akin to the \textit{tabular maximum likelihood} approach. A distinguishement is made between \textit{known} and \textit{unknown} states; a state is \textit{known} after it has been visited an arbitrary number of times, thus its learned dynamics must be close to the true dynamics. If the current state is \textit{unknown}, then \textit{balanced wandering} takes place, and the agent explores by selecting the action that has been taken the least number of times from that state. If the current state is \textit{known} then the action is chosen greedily with respect to the current model.
% \\ R-MAX \cite{10.1162/153244303765208377} is an extension and generalisation of $E^3$. It begins with an optimistic initial model such that every transition leads to some imaginary state that returns the maximal reward, denoted $R_{max}$. R-MAX uses the same concept of \textit{known} and \textit{unknown} states as in $E^3$; after a state becomes known, then the model is updated with the collected statistics regarding that state. The agent always follows the optimal policy according to the model. R-MAX removes the need for explicit contemplation between exploring and exploiting as in $E^3$; exploration and exploitation occur naturally.
% \\ Optimistic Initial Model (OIM) \cite{10.1145/1390156.1390288} begins with an optimistic model, the same as R-MAX. The model is updated through observations similarly to $E^3$. The agent maintains two Q-functions, $Q^r$ and $Q^e$, which are initialised and updated according to observations through dynamic programming. $Q^r$ corresponds to the "real", extrinsic rewards, whereas $Q^e$ corresponds to the intrinsic exploration rewards. Whilst exploring, the agent acts optimally with respect to the combination of the two Q-functions; thus $Q^e$ can be seen as an exploration bonus. During this, the agent will either act near-optimally, or will gain new information it can utilise. After exploration is terminated, $Q^e$ is dropped and the agent acts optimally with respect to $Q^r$.
% \subsection{Intrinsically Motivated}

% \subsection{Constrained}

% \subsection{Our Perspective}

\section{Exploration in Reinforcement Learning}
Within this section we provide a brief survey of exploration methods in RL.  Thrun \cite{Thrun-1992-15850} distinguished between \textit{directed}, where exploration is informed by memory about the state space, and \textit{undirected}, where exploration is driven by randomness. Amin et al. \cite{DBLP:journals/corr/abs-2109-00157} distinguished between reward-free and reward-based exploration, each of which category was then broken down into memory-free (undirected) and memory-based exploration (directed). However it remains that there is no definitive taxonomy for exploration methods. Thus, we choose to distinguish between model-free and model-based methods, akin to undirected and directed methods. The reader should note that not all of the methods we describe are related, or relevant, to our proposed method; however, we want to give a clear background of the topic, and show where our reasoning comes from.

\subsection{Model-Free}
Model-free exploration methods are characterised by the fact that they do not use, or learn, a model aid exploration. This means that they can be robust to varying domains, where designing or learning a model could prove difficult. However, this means that they must rely on other methods to drive exploration, which are grounded in randomness.
\\ Purely random exploration comes in the form of a Random Walk, or unguided random search \cite{anderson86}, which arises from randomly sampling actions with uniform probability. Exploration occurs naturally when the agent moves away from the goal, rather than closer to it. This is perhaps the most naive exploration method, since it is entirely random, and thus it is very inefficient.
\\ Boltzmann exploration samples an action according to the Boltzmann distribution:
\begin{equation}
\label{eqn:boltzmann}
\pi(a|s) = \frac{e^{Q(s,a)/\tau}}{\sum_{a' \in A}e^{Q(s,a')/\tau}}
\end{equation}
Where the temperature, $0 \le \tau \le 1$, is a hyperparameter that determines how frequently random actions are chosen; as $\tau$ approaches 0, the policy approaches greediness \cite{Thrun-1992-15850, DBLP:journals/corr/abs-2109-00157}. 
% Boltzmann exploration can similarly get stuck in a local optima, particularly in domains with sparse rewards.
\\$\epsilon$-greedy \cite{Watkins:1989, conf/nips/Sutton95} uses a hyperparameter, $0 \le \epsilon \le 1$ to balance between exploration and exploitation. With probability $\epsilon$ the agent explores by taking a random action, which is sampled with uniform probability, with probability $1-\epsilon$ the agent exploits by taking the best action, which is selected greedily with respect to the current policy. 

\\Model-free methods are easy to implement, and can be robust to varying domains. However, they are generally inefficient due to their random nature, and lack temporal persistence - although a temporally-persistent form of $\epsilon$-greedy that uses temporally extended sequences of actions was propose by Dabney \cite{dabney2021temporallyextended}. Furthermore, they introduce hyperparameters which may need to be tuned to specific domains; this can be difficult, although simulated annealing (temporal decay) seems to be an effective choice. Somehow, they continue to be ubiquitous in practice, with $\epsilon$-greedy being the most prevalent.

\subsection{Model-Based}
Conversely to model-free exploration methods, model-based methods use, or learn, a model to aid exploration. We separate these methods into two categories: optimistic methods, and intrinsically motivated methods.
\subsubsection{Optimistic}
Optimistic methods utilise the principle of \textit{optimism in the face of uncertainty} (OFU).
\\ Explicit, Explore or Exploit ($E^3$)  \cite{Kearns+Singh:2002} maintains a partial model of the environment through collecting statistics, akin to the \textit{tabular maximum likelihood} approach. A distinguishment is made between \textit{known} and \textit{unknown} states; a state is \textit{known} after it has been visited an arbitrary number of times, thus its learned dynamics must be close to the true dynamics. If the current state is \textit{unknown}, then \textit{balanced wandering} takes place, and the agent explores by selecting the action that has been taken the least number of times from that state. If the current state is \textit{known} then the action is chosen greedily with respect to the current model.
\\ R-MAX \cite{10.1162/153244303765208377} is an extension and generalisation of $E^3$. It begins with an optimistic initial model such that every transition leads to some imaginary state, \textit{The Garden of Eden State}, that returns the maximal reward, denoted $R_{max}$. R-MAX uses the same concept of \textit{known} and \textit{unknown} states as in $E^3$; after a state becomes known, then the model is updated with the collected statistics regarding that state. The agent always follows the optimal policy according to the model. R-MAX removes the need for explicit contemplation between exploring and exploiting as in $E^3$; exploration and exploitation occur naturally.
\\ Optimistic Intial Model (OIM) \cite{10.1145/1390156.1390288} begins with an optimistic model, the same as R-MAX. The model is updated through observations similarly to $E^3$. The agent maintains two Q-functions, $Q^r$ and $Q^e$, which are initialised and updated according to observations through dynamic programming. $Q^r$ corresponds to the "real", extrinsic rewards, whereas $Q^e$ corresponds to the intrinsic exploration rewards. Whilst exploring, the agent acts optimally with respect to the combination of the two Q-functions; thus $Q^e$ can be seen as an exploration bonus. During this, the agent will either act near-optimally, or will gain new information it can utilise. After exploration is terminated, $Q^e$ is dropped and the agent acts optimally with respect to $Q^r$.
\\ $E^3$, R-MAX and OIM are provably able to achieve near-optimal performance in polynomial time in discrete environments. R-MAX and OIM in particular show that optimism is definitely a useful heuristic, but they are very optimistic in perhaps an unrealistic way; an interesting question to ask is whether this unrealistic level of optimism necessary, and if a \textit{reasonable} level of optimism is just as useful. These methods also assume that the model will eventually become correct, which may not be the case in complex tasks, and in the worst case will learn the entire model.
\subsubsection{Intrinsically Motivated}
Intrinsically motivated methods direct exploration towards states that seem \textit{interesting} or \textit{novel} according to some metric, such as uncertainty.
\\ Model-Based Active Exploration (MAX) \cite{DBLP:journals/corr/abs-1810-12162} is purely an exploration framework, which pure exploitation could follow from. An ensemble of dynamics models is maintained and trained on new observations. At each time step a model is sampled from the ensemble, and a one-step policy is derived from the model such that it maximises the intrinsic reward function, which indicates novel states using expected information gain, based on the Jensen-Shannon Divergence.
\\Plan2Explore \cite{plan2explore} separates learning in to two separate phases. During the first phase, the agent learns a global model of the environment, in the form of a latent dynamics model, whilst maintaining and training an ensemble of models on new observations. To explore, the agent trains an exploration policy in the world model that aims to seek out novel states; novelty is defined by estimating latent disagreement in the ensemble of models with the global model. In the second phase, the agent adapts to downstream tasks by using the extrinsic reward signals.
% \\ Go-Explore \cite{goexplore} consists of two phases. In phase one, an \textit{archive} of states is maintained, and trajectories that can be used to reach them, initially this contains only the start state. Exploration is done by sampling a state from the archive, planning to that state (Go) and from there exploring randomly (Explore). The \textit{archive} is updated when novel states are encountered, or when better trajectories to states already within the \textit{archive} are discovered. This repeats until the end of an episode. In phase two, promising trajectories are made more robust to noise using imitation learning. GoExplore does not explicitly define what exploration strategy should be used whilst exploring.
% \cite{goexplore}, PEG \cite{hu2023planning}
\subsection{Hybrid}
Hybrid approaches tend to use a model to influence a form of model-free exploration.
\\ Abbeel et al. \cite{10.1145/1143844.1143845} use an approximate model and a local policy improvement algorithm to determine a  direction for policy improvement, and then executes trajectories in the real environment along that direction, updating the model through experience
\\Domain Approximation for Reinforcement Learning (DARLING) \cite{AIJ16-leonetti} constrains exploration by computing a \textit{partial policy} from the model, and performing model-free RL, in the form of $\epsilon$-greedy, within it; a \textit{partial policy} is a function that maps states to possible actions, and is constructed from a subset of possible plans according to some metric and threshold (for instance all plans that are shorter than 1.5 times the length of the shortest plan). 

% RL is then done inside the partial policy through $\epsilon$-greedy. 


% This leads to constraining exploration to a set of seemingly reasonable states, enabling the agent to overcome inaccuracies in the model, although the model is not explicitly learned. The main benefits of this approach are its ability to incorporate embedded domain knowledge in the form of a model, and its robustness to inaccuracies in said model, whilst also being able to scale to stochastic, continuous domains, such as a real world robotics navigation task. However, it seems that the model must be somewhat correct; as if the optimal policy does not exist in the space of the partial policy, it will never be discovered. Moreover, the use of $\epsilon$-greedy is undesirable, due to the random nature it introduces; however efficiency is not impacted too much, due to the reduced space.
\\ Reward shaping refers to the method of augmenting the reward function in a way that aids in directing the agent towards the goal. Grzes and Kudenko proposed a method \cite{4670492} of creating a reward shaping function from a plan, which aids the agent in learning faster, particularly in domains with sparse reward.
 


\subsection{Conclusions}



% Thrun \cite{Thrun-1992-15850} distinguished between directed and undirected exploration. Direct exploration refers to exploraiton that is informed by memory about the state space, wjereas undirected exploration is uninformed, random.

% \section{Exploration in Reinforcement Learning}
% Thrun \cite{Thrun-1992-15850} distinguished exploration methods in to two main categories: directed and undirected. Directed exploration refers to exploration that is informed by memory about the state space, whereas undirected exploration is uninformed, random. More recently, Amin, et al. \cite{DBLP:journals/corr/abs-2109-00157} presented a comprehensive survey on exploration in RL, which distinguished between reward-free and reward-based exploration; each of these categories is then broken down into memory-free (undirected) and memory-based (directed) exploration.
% % However, since there is not a definitive taxonomy of exploration methods in RL, we choose to distinguish between model-free and model-based exploration methods. Model-free methods tend to be classic, simple methods based, often based on randomness. These methods are not directly relevant to our approach, however its important to discuss them to further motivate our approach.
% % Model-based methods are generally more advanced exploration methods, that may utilise planning, optimism, intrinsic motivation and curiosity to explore. These methods are most relevant to our approach.

% \subsection{Model-Free Exploration}
% Purely random exploration comes in the form of a Random Walk, or unguided random search \cite{anderson86}, which arises from randomly sampling actions with uniform probability, as seen in Equation \ref{eqn:rw}. 
% \begin{equation}
% \label{eqn:rw}
%     a_t \sim P(A)
% \end{equation}
% Where $P(A)$ is the uniform distribution over the action space, $A$. Exploration occurs naturally when the agent moves away from the goal, rather than closer to it, due to the uniform selection of actions. This is perhaps the most naive method of exploration, since it is entirely random and is therefore very inefficient.
% \\ Boltzmann exploration samples an action according to the Boltzmann distribution:
% \begin{equation}
% \label{eqn:boltzmann}
% \pi(a|s) = \frac{e^{Q(s,a)/\tau}}{\sum_{a' \in A}e^{Q(s,a')/\tau}}
% \end{equation}
% Where $\tau$ is the temperature, and determines how frequently random actions are chosen. As $\tau$ decreases, the policy approaches greediness \cite{DBLP:journals/corr/cs-AI-9605103, DBLP:journals/corr/abs-2109-00157}. $\tau$ may be decayed temporally, in the same fashion as $\epsilon$ can be decayed temporally in $\epsilon$-greedy. Here we gave the definition in the context of a $Q$-function, however this can be replaced with any value that estimates the reward received from taking action $a$. The main drawback of this approach is that the initial value for $\tau$ needs to be carefully selected, and may need to be tuned for each environment. This is because as $\tau$ approaches 0 the agent acts greedily, whereas at large values of $\tau$, all actions have an approximately uniform probability of being selected leading to exploration akin to a Random Walk \cite{DBLP:journals/corr/abs-2109-00157}.
% \\$\epsilon$-greedy \cite{Watkins:1989, conf/nips/Sutton95} uses a hyperparameter, $\epsilon$ to balance between exploration and exploitation. With probability $\epsilon$ the agent explores by taking a random action, which is sampled with uniform probability, with probability $1-\epsilon$ the agent exploits by taking the best action, which is selected greedily with respect to the current policy; this is summarised in Equation \ref{eqn:egreedy}.
% \begin{equation}
% \label{eqn:egreedy}
% a_t = 
% \begin{cases}
% \pi(s_t) \qquad \qquad \qquad \text{with probability} \quad 1-\epsilon\\
% \text{random action} \qquad  \text{with probability} \quad $\epsilon$
% \end{cases}
% \end{equation}
% Where $0 \le \epsilon \le 1$, and $\pi$ is the current greedy policy.
% Whilst this provides a clear method for balancing exploration and exploitation, it lacks temporal persistence; although $\epsilon z$-greedy \cite{dabney2021temporallyextended} offered an extension to $\epsilon$-greedy that does use temporal persistance through temporally-extended sequences of actions. Furthermore, by the nature of the name, it is a greedy method which could lead to a local optima. $\epsilon$ may be decayed temporally, to ensure that lots of exploration happens early on, and then exploitation occurs more when learning has occurred. Whilst $\epsilon$-greedy is very easy to implement, it is very inefficient and only converges to optimality in the limit, under certain conditions.
% \\ Whilst these model-free approaches are useful, in that they are generally easy to implement and are robust to varying domains, they tend to share two common drawbacks: practical inefficiency and introducing hyperparameters that have to be tuned. 
% \subsection{Model-Based Exploration}
% The term Model-Based Exploration can be interpreted as a lot of different things. We define Model-Based Exploration as any exploration method that uses a model, known or learned, to direct exploration trajectories.
% \subsubsection{Optimistic}
% \\\textbf{Explicit Explore or Exploit} ($E^3$) \cite{Kearns+Singh:2002} maintains a partial model of the environment through collecting statistics, akin to the \textit{tabular maximum likelihood} approach. A distinguishement is made between \textit{known} and \textit{unknown} states; a state is \textit{known} after it has been visited an arbitrary number of times, thus its learned dynamics must be close to the true dynamics. If the current state is \textit{unknown}, then \textit{balanced wandering} takes place, and the agent explores by selecting the action that has been taken the least number of times from that state. If the current state is \textit{known} then the action is chosen greedily with respect to the current model.
% \\\textbf{R-MAX} \cite{10.1162/153244303765208377} is an extension and generalisation of $E^3$. It begins with an optimistic initial model such that every transition leads to some imaginary state that returns the maximal reward, denoted $R_{max}$. R-MAX uses the same concept of \textit{known} and \textit{unknown} states as in $E^3$; after a state becomes known, then the model is updated with the collected statistics regarding that state. The agent always follows the optimal policy according to the model. R-MAX removes the need for explicit contemplation between exploring and exploiting as in $E^3$; exploration and exploitation occur naturally.
% \\\textbf{Optimistic Initial Model} (OIM) \cite{10.1145/1390156.1390288} begins with an optimistic model, the same as R-MAX. The model is updated through observations similarly to $E^3$. The agent maintains two Q-functions, $Q^r$ and $Q^e$, which are initialised and updated according to observations through dynamic programming. $Q^r$ corresponds to the "real", extrinsic rewards, whereas $Q^e$ corresponds to the intrinsic exploration rewards. Whilst exploring, the agent acts optimally with respect to the combination of the two Q-functions; thus $Q^e$ can be seen as an exploration bonus. During this, the agent will either act near-optimally, or will gain new information it can utilise. After exploration is terminated, $Q^e$ is dropped and the agent acts optimally with respect to $Q^r$.
% \\ $E^3$, R-MAX and OIM are provably able to achieve near-optimal performance in polynomial time in discrete environments. R-MAX and OIM in particular show that optimism is definitely a useful heuristic, but they are very optimistic in perhaps an unrealistic way; an interesting question to ask is whether this unrealistic level of optimism necessary, and if a \textit{reasonable} level of optimism is just as useful. These methods also assume that the model will eventually become correct, which may not be the case in complex tasks, and in the worst case will learn the entire model.
% \subsubsection{Intrinsically Motivated}
% \\\textbf{Model-Based Active Exploration} (MAX) \cite{DBLP:journals/corr/abs-1810-12162} is purely an exploration framework, which pure exploitation could follow from. An ensemble of dynamics models is maintained and trained on new observations. At each time step a model is sampled from the ensemble, and a one-step policy is derived from the model such that it maximises the intrinsic reward function, which indicates novel states using expected information gain, based on the Jensen-Shannon Divergence.
% \\ \textbf{Plan2Explore} \cite{plan2explore} separates learning in to two separate phases. During the first phase, the agent learns a global model of the environment, in the form of a latent dynamics model, whilst maintaining and training an ensemble of models on new observations. To explore, the agent trains an exploration policy in the world model that aims to seek out novel states; novelty is defined by estimating latent disagreement in the ensemble of models with the global model. In the second phase, the agent adapts to downstream tasks by using the extrinsic reward signals.
% \\ \textbf{GoExplore} \cite{goexplore} consists of two phases. In phase one, an \textit{archive} of states is maintained, and trajectories that can be used to reach them, initially this contains only the start state. Exploration is done by sampling a state from the archive, planning to that state (Go) and from there exploring randomly (Explore). The \textit{archive} is updated when novel states are encountered, or when better trajectories to states already within the \textit{archive} are discovered. This repeats until the end of an episode. In phase two, promising trajectories are made more robust to noise using imitation learning. GoExplore does not explicitly define what exploration strategy should be used whilst exploring.
% \\ \textbf{Planning Exploration Goals} (PEG) \cite{hu2023planning} is a modification of GoExplore, where in the first phase, a state is planned to that maximises the estimated exploration value, and exploration ensues from there.


% \\ MAX, Plan2Explore, GoExplore and PEG utilise intrinsic motivation, often determined by estimated uncertainy or information gain, to drive exploration. MAX and Plan2Explore perform task-agnostic exploration, which in theory enables generalisation to downstream tasks. Although, only Plan2Explore was shown to adapt to downstream tasks in high-dimensional, continuous domains, and in-fact it often achieves this in a zero-shot manner. However, this comes at great computational costs, due to their ensemble-based approaches.
% Whilst GoExplore and PEG do not explicitly define exploration strategies, GoExplore was shown to be very effective in difficult exploration tasks using random exploration.

% % \subsection{DARLING}
% Domain Approximation for Reinforcement Learning (DARLING) \cite{AIJ16-leonetti} is neither a model-free or model-based exploration method. It is a hybrid method; it incorporates a model with model-free exploration. A \textit{partial policy} is computed from the model, using the planner; a \textit{partial policy} is a function that maps states to possible actions, and is constructed from a subset of possible plans according to some metric and threshold (for instance all plans that are shorter than 1.5 times the length of the shortest plan). RL is then done inside the partial policy through $\epsilon$-greedy. This leads to constraining exploration to a set of seemingly reasonable states, enabling the agent to overcome inaccuracies in the model, although the model is not explicitly learned. The main benefits of this approach are its ability to incorporate embedded domain knowledge in the form of a model, and its robustness to inaccuracies in said model, whilst also being able to scale to stochastic, continuous domains, such as a real world robotics navigation task. However, it seems that the model must be somewhat correct; as if the optimal policy does not exist in the space of the partial policy, it will never be discovered. Moreover, the use of $\epsilon$-greedy is undesirable, due to the random nature it introduces; however efficiency is not impacted too much, due to the reduced space.